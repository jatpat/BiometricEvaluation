%
% Memory API
%
\chapter{Memory}
\label{chp-memory}
To assist applications with memory management, the Memory package provides
classes to wrap C memory allocations, and other dynamically-sized objects.

\section{AutoBuffer}
\label{sec-autobuffer}
The \lname\ is designed to interoperate with existing C code that has its own
memory management techniques, e.g. \nbis~\cite{nbis}. In these cases, functions
exist to allocate and free blocks of memory, and these calls must be made by
the applications which use those libraries. To assist \sname\ clients that
use these existing libraries, the AutoBuffer class wraps the C memory
management functions, guaranteeing the release of C objects when the
AutoBuffer goes out of scope.

The AutoBuffer constructor takes three function pointers as parameters: one for
C object construction, one for destruction, and a third, optional, function
for copying the C object. If the latter is passed a NULL, the AutoBuffer and
the underlying C object cannot be copied, and an exception will be thrown.

\lstref{autobufferuse} shows the use of AutoBuffer to wrap the memory
allocation routines that are part of the \nbis\ ANSI/NIST library.

\begin{lstlisting}[caption={Using the AutoBuffer}, label=autobufferuse]
#include <be_memory_autobuffer.h>
#include <iostream>
extern "C" {
  #include <an2k.h>
}

int
main(int argc, char* argv[]) {


    /*
     * alloc_ANSI_NIST(), free_ANSI_NIST(), and copy_ANSI_NIST()
     * are functions in the NBIS AN2K library.
     */
    Memory::AutoBuffer<ANSI_NIST> an2k =
        Memory::AutoBuffer<ANSI_NIST>(&alloc_ANSI_NIST,
            &free_ANSI_NIST, &copy_ANSI_NIST);
    if (read_ANSI_NIST(fp, an2k) != 0) {
            cerr << "Could not read AN2K file." << endl;
            return (EXIT_FAILURE);
    }

    for (int i = 1; i < an2k->num_records; i++) {
        // process the ANSI/NIST record ...
    }
}

\end{lstlisting}

\section{AutoArray}
\label{sec-autoarray}
At its simplest level, AutoArray is a C-style array with numerous convenience 
methods, such as being able to query the number of elements.  C++ iterators
can be used over the contents of the array.  The array can be resized without
the need to create a new object.  C++ operator overloading allows AutoArray
objects to be passed to C-style functions that expect pointers to AutoArray's
template type.

AutoArray is used extensively in \sname\ to help eliminate mistakes when
manually allocating memory.  The AutoArray constructor will allocate needed
memory using {\tt new} and the destructor will {\tt delete} it.  This ensures
that any allocated memory will be appropriately freed when the AutoArray goes
out of scope.  Copy constructors and methods as well as the assignment operator
all correctly manage memory so the client does not have to.  Several objects in
\sname\ return AutoArray objects to assist clients in proper memory management.

A common use of AutoArray is to deal with records sequenced from a
RecordStore. \lstref{autoarrayrsuse} demonstrates this.  Notice the
omission of memory management statements -- they are completely unnecessary.

\begin{lstlisting}[caption={Using \nameref{sec-autoarray} with RecordStores}, label=autoarrayrsuse]
#include <be_io_dbrecstore.h>
#include <be_memory_autoarray.h>

#include <iostream>

using namespace BiometricEvaluation;

int
main(
    int argc,
    char *argv[])
{
	IO::DBRecordStore rs("db_recstore", ".", IO::READONLY);

	uint64_t value_size = 0;
	string key("");
	Memory::AutoArray<uint8_t> value;
	for (bool stop = false; stop == false; ) {
		try {
			// Non-destructively resize the AutoArray to hold
			// the next record.
			value.resize(rs.sequence(key, NULL));

			// Read the record into the AutoArray (treats the
			// AutoArray as a pointer).
			rs.read(key, value);

			// Do something with value.
			std::cout << "Key " << key << " has a value of " <<
			    value.size() << " bytes" << std::endl;
		} catch (Error::ObjectDoesNotExist) {
			stop = true;
		}
	}

	return (0);
}
\end{lstlisting}

AutoArray is adapted from "c\_array"~\cite[496]{cpp:plguide}.

\section{IndexedBuffer}
\label{sec-indexedbuffer}
Many applications have a need to read items from a data record and take
action based on the value of the item read. For example, when reading a
biometric data record, the number of finger minutiae points in the record
is indicated by a value in the record header. Furthermore, the record format
may be of a different endianess than the application's host platform.

The IndexedBuffer class is used to access data from a buffer in fixed-size
amounts in sequence. Objects of this class maintain an index into the buffer
as internal state and reads out of the buffer, when using certain methods,
adjust the index. In addition, standard subscript access can be done on
on the buffer (reads and writes) without affecting the index. The basic element
type is an unsigned eight-bit value.
The IndexedBuffer object can be created to either manage the buffer memory
directly, or to ``wrap'' an existing buffer. 

Methods to retrieve elements from the buffer are defined in the class's
interface. These functions are used to retrieve 8/16/32/64-bit values while
moving the internal index. Several functions are also provided to take into
account the endianess of the underlying data.

\lstref{indexedbufferuse} shows how an application can read a data record in
big-endian format.

\begin{lstlisting}[caption={Using the IndexedBuffer}, label=indexedbufferuse]
#include <be_memory_autoarray.h>
#include <be_memory_indexedbuffer.h>

int
main(int argc, char* argv[]) {

	uint64_t size = IO::Utility::getFileSize("BiometricRecord");
	FILE *fp = std::fopen("BiometricRecord", "rb");
	Memory::IndexedBuffer iBuf(size);
	fread(iBuf, 1, size, fp);
	fclose(fp);
	Memory::IndexedBuffer iBuf(recordData, recordData.size());

	uint32_t lval;
        uint16_t sval;

	/*
	 * Record is big-endian:
	 * ---------------------------------
	 * | NAME | LENGTH | ID |    ...   |
	 * ---------------------------------
	 *     4      4      2
	 */

	/* Read a 4-byte C string */
        lval = iBuf.scanU32Val();        /* Format ID */
        char *cptr = (char *)&lval;
        string s(cptr);

	/* Read a 4-byte length */
	lval = iBuf.scanBeU32Val()

	/* Read a 2-byte ID */
	sval = iBuf.scanBeU16Val();
}
\end{lstlisting}
