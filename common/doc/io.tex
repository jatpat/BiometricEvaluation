%
% I/O API
%
\chapter{Input/Output}
\label{chp-io}

The {\em BiometricEvaluation::IO}) classes are used
by applications for the common types of input and output: managing stores of
data, log files, and individual file management. The goal of using the IO API 
is to relieve applications of the need to manage low-level I/O operations such
as file opening, writing, and error handling. Furthermore, by using the classes
defined in {\em IO}, the actual storage mechanism used for data can be managed 
efficiently and placed in a consistent location for all applications.

Many classes manage persistent storage within the file system,
taking care of file open and close operations, as well as error handling. When
errors do occur, exceptions are thrown, which then must be handled by the
application.

\section{Utility}

The {\em IO::Utility} class provides static methods that are used to
manipulate the file system and other low-level mechanisms. These methods
can be used by applications in addition to being used by other classes 
within the Biometric Evaluation framework.

\section{Record Management}

The {\em IO::RecordStore} class provides an abstraction for performing
record-oriented input and output to an underlying storage system. Each
implementation of the {\em RecordStore} provides a self-contained entity to
manage data on behalf of the application in a reliable, efficient manner.

Many biometric evaluations generate thousands of files in the form of processed
images and biometric templates, in addition to consuming large numbers of files
as input. In many file systems, managing large numbers of files in not
efficient, and leads to longer run times as well as difficulties in backing up
and processing these files outside of the actual evaluation.

The {\em RecordStore} abstraction de-couples the application from
the underlying storage, enabling the implementation of different strategies for
data management. One simple strategy is to store each record into a separate
file, reproducing what has typically been done in the evaluation software
itself. Archive files and small databases are other implementation strategies
that have been used.

Use of the {\em RecordStore} abstraction allows applications to switch
storage strategy by changing a few lines of code. Furthermore, error handling
is consistent for all strategies by the use of common exceptions.

Record stores provide no semantic meaning to the nature of the data that passes
through the store. Each record is an opaque object, given to the store as a
pointer and data lengrh, and is associated with a string, the key. Keys must
be unique and are associated with a single record. Attempts to insert multiple
records with the same key result in an exception being thrown.

\section{Logging}
\label{sec-logging}

Many applications are required to log information during their processing. In
particular, the evaluation test drivers often create a log record for each
call to the software under test. There is a need for the log entries to be
consistent, yet any logging facility must be flexible in accepting the type of
data that is to be written to the log file.

The logging classes in the {\em IO} package provide a straight-forward method
for applications to record their progress without the need to manage the
low-level output details.
There are two classes, {\em IO::LogCabinet} and {\em IO::LogSheet} that are used
to perform consistent logging of information by applications. A {\em LogCabinet}
contains a set of {\em LogSheet}s.

A {\em LogSheet} is an output stream (subclass of {\em std::ostringstream}),
and therefore can handle built-in types and any class that supports streaming.
The example code in \ref{logcabinetuse} shows how an application can use a
{\em LogSheet}, contained within a {\em LogCabinet}, to record operational
information.

Log sheets are simple text files, with each entry numbered by the {\em LogSheet}
class when written to the file. The description of the sheet is placed at the
top of the file during construction of the {\em LogSheet} object. A call to the
{\em newEntry()} method commits the current entry to the log file, and resets
the write position to the beginning of the entry buffer.

In addition to streaming by using the {\em LogSheet::<<} operator, applications
can directly commit an entry to the log file by calling the {\em write()}
method, thereby not disrupting the entry that is being formed. After an entry
is committed, the entry number is automatically incremented.

The example in Listing~\ref{logcabinetuse} shows application use of the
logging facility.

\lstset{language=c++}
\begin{lstlisting}[caption={Using a LogSheet within a LogCabinet}, label=logcabinetuse]
#include <be_io_logcabinet.h>
using namespace BiometricEvaluation;
using namespace BiometricEvaluation::IO;

LogCabinet *lc;
try {
    lc = new LogCabinet(lcname, "A Log Cabinet", "");
} catch (Error::ObjectExists &e) {
    cout << "The Log Cabinet already exists." << endl;
    return (-1);
} catch (Error::StrategyError& e) {
    cout << "Caught " << e.getInfo() << endl;
    return (-1);
}
auto_ptr<LogCabinet> alc(lc);
try {
    ls = alc->newLogSheet(lsname, "Log Sheet in Cabinet");
} catch (Error::ObjectExists &e) {
    cout << "The Log Sheet already exists." << endl;
    return (-1);
} catch (Error::StrategyError& e) {
    cout << "Caught " << e.getInfo() << endl;
    return (-1);
}
ls->setAutoSync(true);	// Force write of every entry when finished
int i = ...
*ls << "Adding an integer value " << i << " to the log." << endl;
ls->newEntry();		// Forces the write of the current entry
.........
delete ls;
return;			// The LogCabinet is destructed by the auto_ptr
\end{lstlisting}

