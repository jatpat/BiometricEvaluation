%
% Finger API
%
\chapter{Finger}
\label{chp-finger}
One of the most commonly used biometric source is the fingerprint. Multiple
types of information can be derived from a fingerprint, including minutiae
and the pattern, such as whorl, etc. The \namespace{Finger} package contains the types,
classes, and other items that are related to fingers and fingerprints. Objects
of the \namespace{Finger} classes are typically not used in a stand-alone fashion, but are
usually obtained from an object in the
\namespace{DataInterchage}~(\chpref{chp-datainterchange}) package.

Several enumerated types are defined in the \namespace{Finger} package. The types are used
to represent those elements related to fingers and fingerprints that are common
across all data formats. Types that represent finger position, impression type,
and others are included in the package. Stream operators are defined for these
types so they can be printed in human-readable format.

Most of the classes in the \namespace{Finger} package represent data taken directly from
a record in a standard format (e.g. ANSI/NIST~\cite{std:an2k}). In addition
to general information, such as finger position, other information may be
represented: The source of the finger image; the quality of the image, etc.
In addition to this descriptive information, the finger object will provide
the set of derived minutiae or other data sets.

When representing the information about a finger (and fingerprint), the class
in the \namespace{Finger} package implements the interface defined in the \namespace{View} package.
A finger is a specific type of view in that it represents all the available
information about the finger, including the source image, minutiae (often in
several formats), as well as the capture data (date, location, etc.)

\section{ANSI/NIST Minutiae Data Record}
\label{sec-an2kminutiaedatarecord}
Finger views are objects that represent all the available information for a
specific finger as contained in one or more biometric records. For example,
an ANSI/NIST file may contain a Type-3 record (finger image) and an associated
Type-9 record (finger minutiae). A finger view object based on the ANSI/NIST
record can be instantiated and used by an application to retrieve all the
desired information, including the source finger image. The internals of
record processing and error handling are encapsulated within the class.

The \sname provides several classes that are derived from a base View class,
contained within the \namespace{Finger} package. See~\chpref{chp-finger} for
more information
on the types associated with fingers and fingerprints. This section discusses
finger views, the classes which are derived from the general \class{View} class.
These subclasses represent specific biometric file types, such as ANSI/NIST
or INCITS/M1. In the latter case, two files must be provided when constructing
the object because INCITS finger image and finger minutiae records are defined
in two separate standards.

\subsection{ANSI/NIST Finger Views}

An ANSI/NIST record may contain one or more finger views, each based on a type
of finger image. These Type-3, Type-4, etc. records contain the image and
Type-9 minutiae data, among other information. These record types are grouped
into either the fixed- or variable-resolution categories, and are represented
as specific classes within \sname, \class{AN2KViewFixedResolution} and
\class{AN2KViewVariableResolution}.

The \class{AN2KMinutiaeDataRecord} class represents all of the information taken
from a ANSI/NIST Type-9 record. A Type-9 record may include minutiae data items
in several formats (standard and proprietary) and the impression type code.

\lstref{lst:an2kfingerviewuse} shows how an application can use the
\class{AN2KViewFixedResolution} to retrieve image information, image data, and
derived minutiae information from a file containing an ANSI/NIST record with
Type-3 (fixed resolution image) and Type-9 (fingerprint minutiae) records.

\begin{lstlisting}[caption={Using an AN2K Finger View}, label=lst:an2kfingerviewuse]
#include <fstream>
#include <iostream>
#include <memory>
#include <be_finger_an2kview_fixedres.h>
using namespace std;
using namespace BiometricEvaluation;

int
main(int argc, char* argv[]) {

    Finger::AN2KViewFixedResolution *_an2kv
    try {
        _an2kv = new Finger::AN2KViewFixedResolution("type9-3.an2k",
	    TYPE_3_ID, 1);
    } catch (Error::DataError &e) {
        cerr << "Caught " << e.getInfo()  << endl;
        return (EXIT_FAILURE);
    } catch (Error::FileError& e) {
        cerr << "A file error occurred: " << e.getInfo() << endl;
        return (EXIT_FAILURE);
    }
    std::unique_ptr<Finger::AN2KView> an2kv(_an2kv);

    cout << "Image resolution is " << an2kv->getImageResolution() << endl;
    cout << "Image size is " << an2kv->getImageSize() << endl;
    cout << "Image color depth is " << an2kv->getImageColorDepth() << endl;
    cout << "Compression is " << an2kv->getCompressionAlgorithm() << endl;
    cout << "Scan resolution is " << an2kv->getScanResolution() << endl;

    // Save the finger image to a file.
    tr1::shared_ptr<Image::Image> img = an2kv->getImage();
    if (img.get() == NULL) {
       cerr << "Image was not present." << endl;
       return (EXIT_FAILURE);
    }
    string filename = "rawimg";
    ofstream img_out(filename.c_str(), ofstream::binary);
    img_out.write((char *)&(img->getRawData()[0]),
        img->getRawData().size());
    if (img_out.good())
            cout << "\tFile: " << filename << endl;
    else {
        img_out.close();
        cerr << "Error occurred when writing " << filename << endl;
        return (EXIT_FAILURE);
    }
    img_out.close();

    // Get the finger minutiae sets. AN2K records can have more than one
    // set of minutiae for a finger.

    vector<Finger::AN2KMinutiaeDataRecord> mindata = an2kv->getMinutiaeDataRecordSet();
}
\end{lstlisting}

\subsection{ISO/INCITS Finger Views}

The ISO~\cite{org:iso:sc37} and INCITS~\cite{org:incits} standards typically use
separate files for the source biometric data and the derived data. For example,
the ISO 19794-2 standard is for fingerprint minutiae data, while 19794-4 is for
finger image data. The corresponding \sname view objects are constructed with
both files, although a view can be constructed with only one file. In the
latter case, the view object will represent only that information contained in
the single file.

\lstref{lst:an2kfingerviewuse} shows how an application can create a view from
a ANSI/INCTIS 378 finger minutiae format record~\cite{std:ansi378-2004}.

\begin{lstlisting}[caption={Using an INCITS Finger View}, label=lst:incitsfingerviewuse]
#include <stdlib.h>
#include <fstream>
#include <iostream>
#include <be_finger_ansi2004view.h>
#include <be_feature_incitsminutiae.h>
using namespace std;
using namespace BiometricEvaluation;

int
main(int argc, char* argv[]) {

    Finger::ANSI2004View fngv;
    try {
        fngv = Finger::ANSI2004View("test_data/fmr.ansi2004", "", 3);
    } catch (Error::DataError &e) {
        cerr << "Caught " << e.getInfo()  << endl;
        return (EXIT_FAILURE);
    } catch (Error::FileError& e) {
         cerr << "A file error occurred: " << e.getInfo() << endl;
         return (EXIT_FAILURE);
    }
    cout << "Image resolution is " << fngv.getImageResolution() << endl;
    cout << "Image size is " << fngv.getImageSize() << endl;
    cout << "Image color depth is " << fngv.getImageColorDepth() << endl;
    cout << "Compression is " << fngv.getCompressionAlgorithm() << endl;
    cout << "Scan resolution is " << fngv.getScanResolution() << endl;
    
    Feature::INCITSMinutiae fmd = fngv.getMinutiaeData();
    cout << "Minutiae format is " << fmd.getFormat() << endl;
    Feature::MinutiaPointSet mps = fmd.getMinutiaPoints();
    cout << "There are " << mps.size() << " minutiae points:" << endl;
    for (size_t i = 0; i < mps.size(); i++)
        cout << mps[i];

        Feature::RidgeCountItemSet rcs = fmd.getRidgeCountItems();
    cout << "There are " << rcs.size() << " ridge count items:" << endl;
    for (int i = 0; i < rcs.size(); i++)
        cout << "\t" << rcs[i];

    Feature::CorePointSet cores = fmd.getCores();
    cout << "There are " << cores.size() << " cores:" << endl;
    for (int i = 0; i < cores.size(); i++)
        cout << "\t" << cores[i];

    Feature::DeltaPointSet deltas = fmd.getDeltas();
    cout << "There are " << deltas.size() << " deltas:" << endl;
    for (int i = 0; i < deltas.size(); i++)
        cout << "\t" << deltas[i];

    exit (EXIT_SUCCESS);
}
\end{lstlisting}
