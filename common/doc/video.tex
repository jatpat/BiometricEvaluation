%
% Video API
%
\chapter{Video}
\label{chp-video}

The \namespace{Video} package is used to access video (and, in the future,
audio) streams from containers in several formats, such as MPEG4. The classes
in this package rely on the FFmpeg~\cite{libffmpeg} libraries to de-multiplex
video streams from a container, and to decode the streams and retrieve the
frames from the video.

\section{Container}
\label{sec-videocontainer}

\class{Container} objects can be instantiated in three ways:
\begin{enumerate}
\item With a filename: Memory usage will equal to the size of the container
stream;
\item With a \class{AutoArray::uint8Array}: Memory usage will be twice that of
the size of the container stream;
\item With a \code{std::shared\_ptr} wrapping a \class{AutoArray::uint8Array}:
Memory usage equal to the size of the container stream. Applications
must not modify the container data.
\end{enumerate}

By careful coding, the application can prevent duplicate copies of the container
buffer when using method three. By taking advantage of \CppXI move semantics,
\sname and the application avoid duplicate copies.
See~\lstref{lst:videouse} for examples of using all three methods.

\section{Stream}
\label{sec-videostream}

\class{Stream} objects represent a single video stream within the container
and provide access to individual frames from the video stream. In addition,
these frames can be retrieved at their native size, or can be scaled to a
different size. Frames can be returned as 24-bit red/green/blue images,
grayscale, or two-color monochrome.

\class{Stream} objects can be obtained only from a \class{Container} object.
The reason for this is that video frames must be pulled from a stream that
is de-multiplexed from the container stream shared with the \class{Container}
object. Future versions of \sname may allow for \class{Stream}s to be directly
instantiated with coded video streams.

\lstref{lst:videouse} shows the use of \class{Container} and \class{Stream}.

\begin{lstlisting}[caption={Using the Video Framework}, label=lst:videouse]
#include <iostream>
#include <be_memory_autoarray.h>
#include <be_io_utility.h
#include <be_video_container.h>

using namespace BiometricEvaluation;
using namespace std;

int
main(int argc, char* argv[])
{
        std::unique_ptr<Video::Container> pvc;

        std::string filename = "./test_data/2video1audio.mp4";
        if ((argc != 1) && (argc != 2)) {
                cerr << "usage: " << argv[0] << " [filename]" << endl
                    << "If <filename> is not given, " << filename
                    << " is used instead." << endl;
                return (EXIT_FAILURE);
        }
        if (argc == 2)
                filename = argv[1];

        cout << "Construct an program stream from file "
            << filename << endl;
	/*
	 * Three ways to open the container:
	 * 1) Have the framework open the file directly;
	 * 2) Read the file into a local buffer and give that to the framework;
	 * 3) Read the file into a buffer wrapped in a shared pointer and pass
	 *    that to the framework.
	 */
        try {
//              pvc.reset(new
//                  Video::Container(filename));

//              Memory::uint8Array buf =
//                  IO::Utility::readFile(filename);
//              pvc.reset(new Video::Container(buf));

                std::shared_ptr<Memory::uint8Array> buf;
                buf.reset(new Memory::uint8Array(
                    IO::Utility::readFile(filename)));
                pvc.reset(new Video::Container(buf));
        } catch (Error::Exception &e) {
                cout << "Caught: " << e.whatString() << endl;
                return (EXIT_FAILURE);
        }

        cout << "Video Count: " << pvc->getVideoCount() << endl;

        std::unique_ptr<Video::Stream> stream;
	/*
	 * Open the first video stream.
	 */
        try {
                stream = pvc->getVideoStream(1);
        } catch (Error::Exception &e) {
                cerr << "Could not retrieve video stream: " << e.whatString()
                    << endl;
                return (EXIT_FAILURE);
        }
        /*
         * Read all the frames, one at a time, scaled down and converted
	 * to 8-bit grayscale.
         */
        float scaleFactor = 0.5;
        Image::PixelFormat pixelFormat = Image::PixelFormat::Gray8;
        stream->setFrameScale(scaleFactor, scaleFactor);
        stream->setFramePixelFormat(pixelFormat);
        uint64_t expectedCount = stream->getFrameCount();

        cout << "First video stream: " << stream->getFPS() << " FPS, "
            << expectedCount << " frames." << endl;
        /*
         * The frame count can be zero, meaning unknown. If that is the case,
	 * loop until a parameter error is indicated.
         */
        if (expectedCount == 0)
		expectedCount = 99999999;
	uint64_t count = 0;
        for (uint64_t f = 1; f <= expectedCount; f++) {
                try {
                        auto frame = stream->getFrame(f);
                        count++;
			/* Do something with frame.data */
			std::cout << "frame size is "
			    << frame.size.xSize << "x" << frame.size.ySize
			    << std::endl;
                } catch (Error::ParameterError &e) {
                        cout << "No more frames.";
                        break;
                } catch (Error::Exception &e) {
                        std::cout << "Caught " << e.whatString() << endl;
                        return (EXIT_FAILURE);
                }
        }
        cout << "Retrieved " << count << " frames." << endl;
        return (EXIT_SUCCESS);
}
\end{lstlisting}

