%
% Messaging
%
\chapter{Messaging}
\label{chp_messaging}

\lname contains a collection of classes to facilitate reciving messages 
asynchronously over a network. What is done with these messages and how (or if)
to respond is ultimately up to the application. \sname uses this messaging in
a concrete way to receive text-based commands from a \code{telnet} session
over the Internet.

\section{Message Center}
\label{sec_mesaging_message-center}

\class{Process::MessageCenter} is the public-facing class
an application uses to receive messages over a network. A \textit{message} 
is a user-defined blob of data stored in an array of bytes.  Instantiate
a \class{MessageCenter}, and it will dilligently await connections on the
specified port in a separate process. During its run-loop, the appplication
may poll or wait to determine if a message is waiting. The application has
the choice of dealing with the message, sending a response, or ignoring the
message entirely. Because the \class{MessageCenterListener} is in a separate
process, the main run-loop of the application does not have to be interrupted.
The \class{MessageCenter} classes utilize existing framework inter-process
communication techniques to propagate messages (see
\subsecref{subsec:communications}).

\begin{lstlisting}[caption={Basic \class{MessageCenter} Usage}, label=lst:message-center]
namespace BE = BiometricEvaluation;

uint32_t clientID;
BE::Memory::uint8Array message;
BE::Process::MessageCenter mc;
for (;;) {
	/* ... do work ... */

	if (mc.hasUnseenMessages()) {
		mc.getNextMessage(clientID, message);
		std::cout << clientID << " sent a " << message.size() <<
		    " byte message." << std::endl;

		Memory::AutoArrayUtility::setString(message, "ACK\n");
		mc.sendResponse(clientID, message);
	}
}
\end{lstlisting}

Messages can be sent to the \class{MessageCenter} in a number of ways, like
\code{telnet} connections or \code{write()}ing to a socket. Messages are
terminated with a newline (\code{\textbackslash n}) character.

\section{Command Center}
\label{sec_messaging_command-center}

It's easy to see how \class{MessageCenter} might be used for passing
\textit{commands} to a running application. One might want to query the
\textit{status} of an operation or ask a process to \textit{stop}. The aim of
\class{CommandCenter} was to take this common command-passing pattern and
make it easier.

With \class{CommandCenter}, an application defines one or more
\code{enum class}es using \namespace{Framework::\allowbreak Enumeration}s (see
\secref{sec_framework-enumerations}).  For convenience, the application 
should subclass the \class{Command\allowbreak Parser} \code{template}, with the
enumeration as the templated type. The base class
instantiates a \class{Message\allowbreak Center} and listens for connections.
Just like
\class{MessageCenter}, commands do not have to be dealt with or responded to,
and the application will only know if a command is awaiting a response if 
the application asks. 

Because \class{CommandParser} operates off of strongly-typed enumerations,
a pure virtual method, \code{parse(\allowbreak Command)}, must be implemented
in the  child class. It is expected that this method will simply be a
\code{switch}
statement of all possible enumerations (\textit{commands}). The body of 
the \code{switch} will likely call other methods, each dealing with a single
command.

\class{CommandParser} performs some additional convenience functions to help
application developers quickly respond to commands. A \textit{usage} string
may be automatically sent when an invalid command is received. The
application's main run-loop will never see the failed command attempt. If a
valid command is received, \class{CommandParser} will tokenize any extra text
in the sent command and store it in an easily retrieved \class{vector}.
The method called from \code{parse()} can then sanity-check the
arguments and send an error message back to the client if the arguments are
invalid.

\begin{lstlisting}[caption={Basic \class{CommandCenter} Usage}, label=lst:comand-center]
namespace BE = BiometricEvaluation;

enum class TestCommand
{
	Stop,
	Help
};

template<>
const std::map<TestCommand, std::string>
BE::Framework::EnumerationFunctions<TestCommand>::enumToStringMap {
	{TestCommand::Stop, "STOP"},
	{TestCOmmand::Help, "HELP"}
};

class TestCommandParser : public BE::Process::CommandParser<TestCommand>
{
public:
	void
	parse(
	    const BE::Process::CommandParser<TestCommand>::Command &command)
	{
		switch (command.command) {
		case TestCommand::Stop:
			this->stop(command);
			break;
		case TestCommand::Help:
			this->help(command);
			break;
		}
	}

private:
	void
	stop(
	    const BE::Process::CommandParser<TestCommand>::Command &command)
	{
		/* Ensure proper arguments */
		if (command.arguments.size() != 1) {
			this->sendResponse(command.clientID, "Usage: " +
			    to_string(command.command) + " <process>");
			return;
		}

		/* ... perform stop operation ... */
	}

	void
	help(
	    const BE::Process::CommandParser<TestCommand>::Command &command)
	{
		this->sendResponse(command.clientID, "Available Commands:\n"
		    "\tSTOP <process>\n\tHELP");
	}
};

int
main()
{
	TestCommandParser commandCenter;
	TestCommandParser::Command command;
	for (;;) {
		/* ... do work ... */
		
		if (commandCenter.hasPendingCommands()) {
			commandCenter.getNextCommand(command);
			commandCenter.parse();
		}
	}

	return (EXIT_SUCCESS);
}
\end{lstlisting}


It's perfectly acceptible for an application to make use of more than one
\code{CommandParser} for different \code{enum}s, assuming they are listening on 
different ports.

