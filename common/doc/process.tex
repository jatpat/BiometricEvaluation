%
% Process API
%
\chapter{Process Information}
\label{chp-process}

The \namespace{Process} package is a set of APIs used to gather information on a process,
limit the capabilities of a process, and create manage processes.

\section{Process Statistics}
\label{sec-process_statistics}

When a application is running, there is a need to obtain information of the
process executing that application. The Process API can be used by the
application itself to gather statistics related to the current amount of memory
being used, the number of threads, and other items. Biometric evaluation test
drivers are linked against a third party library, and therefore, the application
writer does not control the thread count or memory usage for much of the
processing. \lstref{processstatisticsuse} shows how an application can
use the \class{Statistics} API.

\begin{lstlisting}[caption={Gathering Process Statistics}, label=processstatisticsuse]
#include <be_error_exception.h>
#include <be_process_statistics.h>
using namespace BiometricEvaluation;

int main(int argc, char *argv[])
{
    Process::Statistics stats;
    uint64_t userstart, userend;
    uint64_t systemstart, systemend;
    uint64_t diff;
    try {
	stats.getCPUTimes(&userstart, &systemstart);

	// Do some long processing....

	stats.getCPUTimes(&userend, &systemend);
	diff = userend - userstart;
	cout << "User time elapsed is " << diff << endl;
	diff = systemend - systemstart;
	cout << "System time elapsed is " << diff << endl;
    } catch (Error::Exception) {
	cout << "Caught " << e.getInfo() << endl;
    }

}
\end{lstlisting}

In addition to using the \namespace{Process} API to gather statistics to be returned from
the function call, the API provides a means to have a ``standard'' set of
statistics logged either synchronously or asynchronously to a 
\class{LogSheet} (See~\secref{sec-logging}) contained within a \class{LogCabinet}.
Applications can
start and stop logging at will to this \class{LogSheet}. Post-mortem analysis can
then be done on the entries in the \class{LogSheet}. \lstref{processstatisticslogging}
shows the use of logging.

The \class{LogSheet} will have a file name constructed from the process name (i.e.
the application executable) and the process ID. An example \class{LogSheet} contains
this information at the start:

\begin{verbatim}
Description: Statistics for test_be_process_statistics (PID 28370)
# Entry Usertime Systime RSS VMSize VMPeak VMData VMStack Threads
E0000000001 728889 6998 1788 57472 62612 31020 84 1
E0000000002 1300802 6998 1792 57472 62612 31020 84 1
\end{verbatim}

The \class{Statistics} object creates the \class{LogSheet} with an appropriate description
and comment entry with column headers. Each gathering of the statistics results
in a single log entry.

\begin{lstlisting}[caption={Logging Process Statistics}, label=processstatisticslogging]
#include <be_error_exception.h>
#include <be_io_logcabinet.h>
#include <be_process_statistics.h>
using namespace BiometricEvaluation;

int main(int argc, char *argv[])
{
    IO::LogCabinet lc("statLogCabinet", "Cabinet for Statistics", "");

    Process::Statistics *logstats;
    try {
	logstats = new Process::Statistics(&lc);
    } catch (Error::Exception &e) {
	cout << "Caught " << e.getInfo() << endl;
	return (EXIT_FAILURE);
    }
    try {
	while (some_processing_to_do) {
	    // Do the work
	    // Synchronously log after the work is done.
	    logstats->logStats();
	}
    } catch (Error::Exception &e) {
	cout << "Caught " << e.getInfo() << endl;
	delete logstats;
	return (EXIT_FAILURE);
    }

    // Set up asynchronous logging, every second
    try {
	logstats->startAutoLogging(1);
    } catch (Error::ObjectExists &e) {
	cout << "Caught " << e.getInfo() << endl;
	delete logstats;
	return (EXIT_FAILURE);
    }

    // Do some other work

    // Stop logging
    logstats->stopAutoLogging();
    delete logstats;
}
\end{lstlisting}

\section{Process Management}
\label{sec-process_management}

During a biometric evaluation or other long-running CPU-bound task, it's 
beneficial to make efficient use of all the hardware available on the system.
If your application is running on a multi-core machine, why not make use of
more than one core?  \sname\ aims to simply this by abstracting the usage of
\code{fork(2)} and \lib{libpthread} to run multiple instances of the same
function simultaneously.

\subsection{Manager}
\label{subsec-process_manager}

There are three class hierarchies involved in the abstraction.  
The \classdoxyref{BiometricEvaluation::Process::Manager} classes control the
technique of process manipulation that will be used.  \sname\ provides two
example abstractions: \class{Fork\allowbreak Manager} and
\class{POSIX\allowbreak Thread\allowbreak Manager}.  When
using \class{Fork\allowbreak Manager}, new processes will be created with 
\code{fork(2)}, with mediated access to these new processes through the
\class{Manager}.  Likewise, \class{POSIX\allowbreak Thread\allowbreak Manager}
creates new POSIX threads.  Because both of theses classes inherit from
\class{Manager}, it is as trivial as changing the \class{Manager} object type
to change how the workload is parallelized.

\subsection{Worker}
\label{subsec-process_worker}

In the application using a \class{Manager}, a \class{Worker} subclass must be
implemented.  An example \class{Worker} is shown in \lstref{process_worker-example}.
The entry-point for a \class{Worker} is the \code{worker\allowbreak Main()} method,
which must be implemented by the client application.  Although
\code{worker\allowbreak Main()} takes no arguments, data may be transmitted
into the object through
\class{Worker\allowbreak Controller}'s~(\ref{subsec-process_workercontroller})
\code{set\allowbreak Parameter()} method.  Within the \class{Worker} instance,
the parameters are then retrieved with \code{get\allowbreak Parameter()} when
provided with the unique parameter name.

A responsible \class{Worker} performs its operations as fast as it can,
however, at any given time, the \class{Manager} may ask the \class{Worker} to
stop.  It then becomes the {\em responsibility of the \class{Worker}} to 
stop as soon as possible.  The \class{Worker} is notified of the stop request
through its \code{stop\allowbreak Requested()} method.  Note that the
\class{Manager} does \textbf{not} force the \class{Worker} to stop, though
prolonged work or cleanup in the \class{Worker} would likely produce
undesired results in the client application.  As such, a responsible
\class{Worker} checkpoints itself to prepare for premature stops requested by
the \class{Manager}.  While it is important for \class{Worker}s to stop as soon
as possible after the request is received, it is also important not to leave
work in an unsynchronized state.  In \lstref{process_worker-example}, notice how
the \class{Employee} must continue the interaction with the \class{Customer}
before a stop request is handled, even if the \class{Employee}'s shift has
ended.  Leaving the method before the \class{Customer}'s order has been
delivered would leave the \class{Customer} object in an unsafe state (hungry). 

\begin{lstlisting}[caption={A Responsible \class{Worker} Implementation}, label=process_worker-example]
#include <cstdlib>
#include <tr1/memory>
#include <queue>

#include <restaurant.h>

#include <be_process_forkmanager.h>

using namespace std;
using namespace BiometricEvaluation;
using namespace Restaurant;

class ResponsibleEmployeeTask : public Process::Worker
{
public:
	int32_t
	workerMain()
	{
		int32_t status = EXIT_FAILURE;
		
		/* Retrieve objects assigned to this Task */
		tr1::shared_ptr<Employee> employee =
		    tr1::static_pointer_cast<Employee>(
		    this->getParameter("employee"));
		tr1::shared_ptr< queue<Customer*> > customers = 
		    tr1::static_pointer_cast< queue<Customer*> >(
		    this->getParameter("customers")
		
		employee->clockIn();
		
		Customer *customer;
		/* Checkpoint after each customer */
		while (this->stopRequested() == false ||
		    employee->isShiftOver() == false) {
			customer = customers->front();
			
			if (customer != NULL) {
				employee->takeOrder(customer);
				employee->cookFood(customer);
				employee->deliverOrder(customer);
				
				customers->pop();
			}
		}
		
		employee->settleCashDrawer();
		employee->clockOut();
		
		status = EXIT_SUCCESS;
		return (status);
	}
	~ResponsibleEmployeeTask() {}
};
\end{lstlisting}

After a \class{Manager} starts its \class{Worker}s, the \class{Manager} has the
option of waiting until all \class{Worker}s exit \code{worker\allowbreak Main()}
before continuing code execution.  If not waiting,
there are several methods the \class{Manager} can perform to keep track of the status of the
\class{Worker}s.  Even if not waiting for \class{Worker}s to return,
a responsible \class{Manager} will wait a reasonable amount of time for
\class{Worker}s to \code{return} before application termination.  An example of this
reasonable waiting period can be seen in \lstref{process_manager-example}.

\subsection{WorkerController}
\label{subsec-process_workercontroller}

The final piece of the process management puzzle is the
\class{Worker\allowbreak Controller} hierarchy.  This class decorates and
mediates communication between the \class{Manager} and the \class{Worker}.  
\class{Worker\allowbreak Controller} objects may only be instantiated by
a \class{Manager} object.  All communications to the \class{Worker}
(e.g. \code{is\allowbreak Working()}) should be delegated through the
\class{Worker\allowbreak Controller}.  If defining a new \class{Manager}, note
that the \class{Worker\allowbreak Controller} may seem unnecessary for the
parallelization technique being employed.  It's true
that some parallelization techniques may not require this ``middle-man''
approach, but others do.  Do not be concerned if a \class{Worker\allowbreak
Controller} implementation ends up being nothing more than a ``pass-thru'' to
the \class{Worker}.

\lstref{process_manager-example} is a continuation of
\lstref{process_worker-example} demonstraiting the use of \class{Manager}s and
\class{Worker\allowbreak Controller}s.

\begin{lstlisting}[caption={Using \class{Manager}s and \class{WorkerController}s}, label=process_manager-example]
int
main(
    int argc,
    char *argv[])
{
	static const uint32_t numEmployees = 3;
	int status = EXIT_FAILURE;
	
	tr1::shared_ptr<Process::Manager> shiftLeader(new Process::ForkManager);
	queue<Customer*> *customers = new queue<Customer*>();
	
	/* Create Employees (Workers/WorkerControllers) */
	tr1::shared_ptr<Process::WorkerController> employees[numEmployees];
	for (uint32_t i = 0; i < numEmployees; i++) {
		employees[i] = shiftLeader->addWorker(
		    tr1::shared_ptr<ResponsibleEmployeeTask>(
		    new ResponsibleEmployeeTask()));
		
		/* Assign employees to each Task */
		employees[i]->setParameter("employee",
		    tr1::shared_ptr<Employee>(new Employee()));
		employees[i]->setParameter("customers",
		    tr1::shared_ptr< queue<Customer*> >(customers);
	}
	
	/* Employees start serving customers while shift leader manages */
	shiftLeader->startWorkers(false);
	
	/* Customers enter the queue ... */
	queue<Restaurant::AdministrativeTasks> adminTasks;
	adminTasks.push("Inventory");
	adminTasks.push("Customer Complaints");
	adminTasks.push("Clean Dining Room");
	
	while (shiftLeader->getNumActiveWorkers() != 0) {
		shiftLeader->doTask(adminTasks.front());
		adminTasks.pop();
	}
	
	/* ...end of the day */
	for (uint32_t i = 0; i < numEmployees; i++)
		if (employees[i]->isWorking()
			shiftLeader->stopWorker(employees[i]);
	
	/* 
	 * Wait a reasonable amount of time before locking up for the night
	 * (in this case, indefinitely).
	 */
	while (shiftLeader->getNumActiveWorkers() > 0)
		sleep(1);
		
	shiftLeader->armAlarmAndExit();
	
	status = EXIT_SUCCESS;	
	return (status);
}

\end{lstlisting}
