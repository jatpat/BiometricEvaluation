%
% Process API
%
\chapter{Process Information}
\label{chp-process}

The Process package is a set of APIs used to gather information on a process,
or to limit the capabilities of a process.

\section{Process Statistics}
\label{sec-process_statistics}

When a application is running, there is a need to obtain information of the
process executing that application. The Process API can be used by the
application itself to gather statistics related to the current amount of memory
being used, the number of threads, and other items. Biometric evaluation test
drivers are linked against a third party library, and therefore, the application
writer does not control the thread count or memory usage for much of the
processing. \lstref{processstatisticsuse} shows how an application can
use the Statistics API.

\lstset{language=c++}
\begin{lstlisting}[caption={Gathering Process Statistics}, label=processstatisticsuse]
#include <be_error_exception.h>
#include <be_process_statistics.h>
using namespace BiometricEvaluation;

int main(int argc, char *argv[])
{
    Process::Statistics stats;
    uint64_t userstart, userend;
    uint64_t systemstart, systemend;
    uint64_t diff;
    try {
	stats.getCPUTimes(&userstart, &systemstart);

	// Do some long processing....

	stats.getCPUTimes(&userend, &systemend);
	diff = userend - userstart;
	cout << "User time elapsed is " << diff << endl;
	diff = systemend - systemstart;
	cout << "System time elapsed is " << diff << endl;
    } catch (Error::Exception) {
	cout << "Caught " << e.getInfo() << endl;
    }

}
\end{lstlisting}

In addition to using the Process API to gather statistics to be returned from
the function call, the API provides a means to have a ``standard'' set of
statistics logged either synchronously or asynchronously to a 
LogSheet (See~\secref{sec-logging}) contained within a LogCabinet.
Applications can
start and stop logging at will to this LogSheet. Post-postmortem analysis can
then be done on the entries in the LogSheet. \lstref{processstatisticslogging}
shows the use of logging.

The LogSheet will have a file name constructed from the process name (i.e.
the application executable) and the process ID. An example LogSheet contains
this information at the start:

\begin{verbatim}
Description: Statistics for test_be_process_statistics (PID 28370)
# Entry Usertime Systime RSS VMSize VMPeak VMData VMStack Threads
E0000000001 728889 6998 1788 57472 62612 31020 84 1
E0000000002 1300802 6998 1792 57472 62612 31020 84 1
\end{verbatim}

The Statistics object creates the LogSheet with an appropriate description
and comment entry with column headers. Each gathering of the statistics results
in a single log entry.

\lstset{language=c++}
\begin{lstlisting}[caption={Logging Process Statistics}, label=processstatisticslogging]
#include <be_error_exception.h>
#include <be_io_logcabinet.h>
#include <be_process_statistics.h>
using namespace BiometricEvaluation;

int main(int argc, char *argv[])
{
    IO::LogCabinet lc("statLogCabinet", "Cabinet for Statistics", "");

    Process::Statistics *logstats;
    try {
	logstats = new Process::Statistics(&lc);
    } catch (Error::Exception &e) {
	cout << "Caught " << e.getInfo() << endl;
	return (EXIT_FAILURE);
    }
    try {
	while (some_processing_to_do) {
	    // Do the work
	    // Synchronously log after the work is done.
	    logstats->logStats();
	}
    } catch (Error::Exception &e) {
	cout << "Caught " << e.getInfo() << endl;
	delete logstats;
	return (EXIT_FAILURE);
    }

    // Set up asynchronous logging, every second
    try {
	logstats->startAutoLogging(1);
    } catch (Error::ObjectExists &e) {
	cout << "Caught " << e.getInfo() << endl;
	delete logstats;
	return (EXIT_FAILURE);
    }

    // Do some other work

    // Stop logging
    logstats->stopAutoLogging();
    delete logstats;
}
\end{lstlisting}
