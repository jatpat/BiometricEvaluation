%
% Image API
%
\chapter{Image}
\label{chp-image}

The Image package maintains the classes and other information related to images
and image processing. Within the \lname, many classes refer to images, such as
when dealing with fingerprint data. Many biometric data standards supply the
actual image in several forms, either compressed or not. Applications can
retrieve the image as stored in the record, or decompressed by the Image class
into a ``raw'' format. Therefore, within the \sname\ several of the common
compression formats are supported, removing the need for applications to
decompress the image directly, while maintaining access to the as-recorded
image format.

\section{The Image Namespace}
\label{sec-imagenamespace}
The Image namespace contains several enumerations and types used to represent
aspects of an image. Compression algorithm and resolution scale are examples
of enumerations and represent some information about the image. Types include
structures to contain a coordinate on the image, or the image size. The
\sname\ does not perform any translation (within the Image classes) of scale
units, or sizing, however, as each set of attributes is copied directly from
the image itself, or the biometric record that contains the image.

The data types containing size and other information are used in the Image
classes, as well as other classes, such as finger views, because those classes
manage information taken directly from the biometric record. In many cases, the
source image attributes are part of the biometric record. Applications can
compare those values against those directly from the Image object, as in most
cases those values are taken directly from the underlying image data.
See~\chpref{chp-view} for more information on image-based biometric records.

The Image namespace contains all of the Image classes that are used to
represent an image in raw or compressed form. These classes are described in
the following sections.

\section{The Image Class}
\label{sec-imageclass}
The Image class is a base class that defines a set of minimum functionality for
all supported image formats.  Once an Image has been constructed, it may not
be changed.  For any supported image format, the following information is
required to be accessible:
\begin{itemize}
\item As-recorded format binary data
\item Compression algorithm
\item Decompressed ("raw") format binary data, grayscale/full color
\item Depth
\item Dimensions, width/height
\item Resolution
\end{itemize}
A rudimentary implementation of generating a grayscale image is provided by the
Image class.  This implementation calculates the luminance value Y (of YCbCr)
for each pixel of a color image.  Image subclasses may implement their own
grayscale conversion methods.

\section{Raw Image}
\label{sec-rawimage}
The Raw Image class represents a decompressed image.  Raw Image has no special 
implementation or additional methods.

\section{JPEG Image}
\label{sec-jpegimage}
The JPEG Image class represents an image encoded according to the JPEG image 
standard~\cite{jpeg}.  Decompression and grayscale conversion are accomplished
via libjpeg~\cite{libjpeg}.

As of version 8.0, libjpeg provided a way to handle JPEG images existing within
in-memory buffers, as opposed to on-disk files.  Because the Image class
requires in-memory buffers, JPEG Image includes a JPEG memory source manager
implementation, but it is built only if a version of libjpeg older than 8.0
is detected.

JPEG Image provides a static function to determine whether or not a
data buffer appears to be encoded in the JPEG image standard format.  Errors
within libjpeg will be caught and rethrown as Exceptions.
\section{JPEG2000 Image}
\label{sec-jpeg2000image}
The JPEG2000 Image class provides Image class functionality to JPEG 2000-encoded
images~\cite{jpeg2000}.  The class makes an attempt to support the following
JPEG 2000 codecs:
\begin{itemize}
\item JPEG 2000 codestream (.J2K)
\item JPEG 2000 compressed image data (.JP2)
\item JPEG 2000 interactive protocol (.JPT)
\end{itemize}
Decompression is provided by the OpenJPEG library (libopenjpeg)
~\cite{libopenjpeg}.  JPEG2000 Image also provides a static function to test
whether or not an image appears to be JPEG 2000-encoded.

Not all information required by the Image class is present in a JPEG 
2000-encoded image.  In particular, some codecs and encoders omit the Display
Resolution Box.  It is generally accepted that the resolution will be 72 pixels
per inch when the Display Resolution Box is not present.

Errors within libopenjpeg will be caught and rethrown as Exceptions.

\section{PNG Image}
\label{sec-pngimage}
The PNG Image class represents an image encoded according to the PNG image 
standard~\cite{png}.  Decompression is provided by libpng~\cite{libpng}.

PNG Image provides a static function to test whether or not an image appears
to be encoded in the PNG image standard format.  Errors within libpng are
caught and rethrown as Exceptions.

\section{WSQ Image}
\label{sec-wsqimage}
Images encoded in the WSQ-image standard~\cite{wsq} are represented by the WSQ
Image class.  The WSQ decompressor found in NBIS~\cite{nbis} is used by this
class.  The class provides a static function to determine whether or not an
image appears to be encoded in the WSQ format.

Errors from the NBIS libwsq will propagate through stderr and will not be
rethrown as exceptions.
