%
% Time API
%
\chapter{Time and Timing}
\label{chp-time}

The \namespace{Time} package within the \lname\ provides a set of classes for performing
timing-related operations, such as elapsed time and limiting execution time.

\section{Elapsed Time}

The \class{Timer} class provides applications a method to determine how long
a block of code takes to execute. On many systems (e.g. Linux) the timer
resolution is in microseconds.

\lstref{lst:timeruse} shows how an application can use a \class{Timer}
object to limit obtain the amount of time used for the execution of a block
of code.

\begin{lstlisting}[caption={Using the \class{Timer}}, label=lst:timeruse]
#include <be_time_timer.h>

int main(int argc, char *argv[])
{
	Time::Timer timer = new Time::Timer();

	try {
		atimer->start();
		// do something useful, or not
		atimer->stop();
		cout << "Elapsed time: " << atimer->elapsed() << endl;
	} catch (Error::StrategyError &e) {
		cout << "Failed to create timer." << endl;
	}
}
\end{lstlisting}

\section{Limiting Execution Time}

The \class{Watchdog} class allows applications to control the amount of time
that a block of code has to execute. The time can be {\em real} (i.e. ``wall'')
time, or {\em process} time (not available on Windows). One typical usage for
a \class{Watchdog} timer is when a call is made to a function that may never return,
due to problems processing an input biometric image.

\class{Watchdog} timers can be used in conjunction with \class{SignalManager} in
order to both limit the processing time of a call, and handle all signals
generated as a result of that call. See~\ref{sec-signalhandling} for
information on the \class{SignalManager} class.

One restriction on the use of \class{Watchdog} is that the POSIX calls for
signal management (\code{signal(3)}, \code{sigaction(2)}, etc.) cannot be
invoked inside of the \code{WATCHDOG} block. This restriction includes calls to
\code{sleep(3)} because it is based on signal handling as well.

\lstref{lst:watchdoguse} shows how an application can use a \class{Watchdog}
object to limit the about of process time for a block of code.

\begin{lstlisting}[caption={Using the \class{Watchdog}}, label=lst:watchdoguse]
#include <be_time_watchdog.h>
int main(int argc, char *argv[])

    Time::Watchdog theDog = new Time::Watchdog(Time::Watchdog::PROCESSTIME);
    theDog->setInterval(300);	// 300 microseconds

    Time::Timer timer;

    BEGIN_WATCHDOG_BLOCK(theDog, watchdogblock1);
        timer.start();
        // Do something that may take more than 300 usecs
        timer.stop();
        cout << "Total time was " <<  timer.elapsed() << endl;
    END_WATCHDOG_BLOCK(theDog, watchdogblock1);
    if (theDog->expired()) {
        timer.stop();
        cerr << "That took too long." << endl;
    }
{
}
\end{lstlisting}

Within the \class{Watchdog} header file, two macros are defined:
\code{BEGIN\_\allowbreak WATCH\allowbreak DOG\_\allowbreak BLOCK()} and \code{END\_\allowbreak WATCH\allowbreak DOG\_\allowbreak BLOCK()}, each taking
the \class{Watchdog} object and label as parameters. The label must be unique
for each \code{WATCHDOG} block.  The use of these macros greatly simplifies
\class{Watchdog} timers for the application, and it is recommended that applications
use these macros instead of directly invoking the methods of the
\class{Watchdog} class, except for setting the timeout value.

Any processing that is normally done at the end of the \code{WATCHDOG} block must also
be done within the \code{expired()} check
due to the fact that process control jumps to the end of the \code{WATCHDOG} block
in the event of a timeout.
A typical example is the use of the \class{Timer}
object inside a \code{WATCHDOG} block, as the example in \lstref{lst:watchdoguse}
shows. In most cases, however, careful application design can remove the need
for duplicate code. In the example, placing the \class{Timer} \code{start()}/\code{stop()} calls
outside of the \code{WATCHDOG} block simplifies the coding, although the small amount
of time for the \code{WATCHDOG} setup and tear down would be included in the time.
